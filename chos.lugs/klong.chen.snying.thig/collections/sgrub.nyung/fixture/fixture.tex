% copyright (c) 2015 Synrc Research Center

\documentclass[4pt,oneside]{article}
% Copyright (c) 2015 Synrc Research Center

\usepackage[english,russian]{babel}
\usepackage{fontspec}
\usepackage{tablefootnote}
\usepackage{polyglossia}
\usepackage{hyphenat}
\usepackage[top=8mm, bottom=12.4mm,
            inner=5mm,outer=8mm,
            paperwidth=122mm, paperheight=180mm]{geometry}

\fontencoding{T1}
\lefthyphenmin=1
\hyphenpenalty=100
\tolerance=3000
\newcommand{\ru}{ \setdefaultlanguage{russian} \setmainfont{Geometria} }
\newcommand{\ti}{ \setdefaultlanguage{tibetan} \setmainfont{DDC Uchen} }
\newtoggle{russian@scriptlangequal}
\newtoggle{tibetan@scriptlangequal}


\begin{document}
   \thispagestyle{empty}

%\newfontfamily{\cyrillicfont}{Geometria}
%\setdefaultlanguage{tibetan}
%\setmainfont{Geometria}
\newfontfamily{\cyrillicfont}{Geometria}
%\setdefaultlanguage{tibetan}
\setmainfont{Geometria}

\scriptsize
\section*{Рекомендации для ежедневной практики}\\
\vspace{0.8cm}
Последовательность и используемые тексты в случаях, когда для практики есть:\\
\\
\indent а) много времени,\\
\indent б) немного времени,\\
\indent в) мало времени.\\
\\
\\
ПРАКТИКА МЕДИТАЦИИ\\
\\
1. ПРЕДВАРИТЕЛЬНАЯ ЧАСТЬ\\
\\
Молитва об удаче (только утром):\\
\indent а) Тексты 1 и 2, стр. 5-8\\
\indent б) Текст 1, стр. 5-7\\
\indent в) Текст 2, стр. 8\\

\\
\noindent Семистрочная молитва (3 раза)\\
\indent а, б, в) Текст 3, стр. 9\\
\\
Прибежище и бодхичитта\\
\indent а) Тексты 4.1, 4.2 и 4.3. Стр. 10-11\\
\indent б) Тексты 4.2, стр. 10, последние 3 строфы и 4.3, стр. 11\\
\indent в) Тексты 4.2, стр. 10, вторая строфа снизу и 4.3, стр. 11\\
\indent Каждая строфа (четверостишие) повторяется 3 раза.\\
\\
Подношение воды (только утром).\\
\indent Перед изображениями Будды, Падмасамбхавы и Гуру (что есть в наличии)\\
\indent наполнить 1 или 7 чашечек с водой.
        Текст 4.4., стр.12\\
\\
2. ОСНОВНАЯ МЕДИТАЦИЯ (памятование дыхания или безмерная любовь)\\
\\
3. ЗАВЕРШАЮЩАЯ ЧАСТЬ\\

\indent а) Тексты 6.1 – 6.5 утром, 6.1 – 6.6 вечером, стр. 16-22\\
\indent б) Тексты 6.1, 6.2 и 6.4, стр. 16-17 и 19 утром, 6.1, 6.2 и 6.4, 6.5 вечером, стр. 16-17 и 19.\\
\indent в) Текст 6.1, стр. 16, и текст 6.4, стр. 19.\\
\\
ПОДНОШЕНИЕ ПИЩИ (перед едой)\\
\\
\indent Текст 4.5, стр. 13\\
\indent а) Все три строфы\\
\indent б) Первая и третья строфы \\
\indent в) Первая строфа\\
\\
Желательно регулярно делать каждую из практик:\\
\\
РИТУАЛ БУДДЫ ШАКЬЯМУНИ Текст 7, стр. 23-27.\\
ГУРУ-ЙОГА ПАДМАСАМБХАВЫ Текст 8, стр. 28-30

\newpage
   \thispagestyle{empty}

{\centering
\Large
Исправления к книге\\
«Тексты молитв и практик\\
для участников затвора\\
с ламой Пема Дудул»\\
}
\vspace{2cm}
\\
\noindent

\begin{tabular}{ll}

СТР. 9  & После последней строки\\
        & «Молю, приди и благослови.»\\
        & добавить строку:\\
        & «ГУРУ ПАДМА СИДДХИ ХУМ»\\
        \\
        \\
СТР. 18 & Строка 4-я сверху\\
        & Напечатано: «Совершенство»\\
        & Должно быть: «Завершение»\\
        \\
        \\
СТР. 21 & Строка 12-я сверху\\
        & Напечатано: «Совершенства»\\
        & Должно быть: «Завершения»\\
        \\
        \\
СТР. 22 & Строка 3-я сверху\\
        &  Напечатано: «всколыхнуть»\\
        &  Должно быть: «исчерпать»\\
        \\
        \\
СТР. 24 & Строки  8-я и 9-я снизу\\
        & Напечатано:\\
        & «Две руки: левая попирает землю,\\
        & Правая – в жесте равновесия, ноги сложены в (ваджрной) позе.»\\
        & Должно быть:\\
        & «Две руки: правая попирает землю,\\
        & Левая – в жесте равновесия, ноги сложены в (ваджрной) позе.»\\
        \\
        \\
СТР. 29 & После 17-й строки сверху («Молю, приди и благослови.»)\\
        & добавить строку:\\
        & «ГУРУ ПАДМА СИДДХИ ХУМ»\\
        \\
\end{tabular}


\end{document}
