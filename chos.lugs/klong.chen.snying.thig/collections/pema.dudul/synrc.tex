% Copyright (c) 2015 Synrc Research Center

%\usepackage{afterpage}
\usepackage[english,russian]{babel}
%\usepackage{graphicx}
%\usepackage{tocloft}
\usepackage{fontspec}
\usepackage{polyglossia}
\usepackage{hyphenat}
%\usepackage{caption}
%\usepackage[usenames,dvipsnames]{color}
\usepackage[top=18mm, bottom=22.4mm,
            inner=15mm,outer=18mm,
            paperwidth=142mm, paperheight=200mm]{geometry}

\hyphenation{бес-чувстве-нен cуще-ствования
 буду-щем разно-образии Дхар-му наско-лько счастли-вого ниж-них
 сохрани-лось всеведую-щие свер-нуть неза-висимости}

\fontencoding{T1}

%\setlength{\cftsubsecnumwidth}{2.5em}
%\defaultfontfeatures{Ligatures=TeX}

% include image for HeVeA and LaTeX

%\makeatletter
%\def\@seccntformat#1{\llap{\csname the#1\endcsname\hskip0.7em\relax}}
%\makeatother

\newcommand{\includeimage}[2]
{\begin{figure}[h!]
\centering
\includegraphics[width=\textwidth]{#1}
\caption{#2}
\end{figure}
}

\newcommand*{\titlePRAYER}
{
\newfontfamily{\cyrillicfont}{Geometria}
\setdefaultlanguage{tibetan}
\setmainfont{Geometria}
    \begingroup
        \thispagestyle{empty}
        \hspace*{0.15\textwidth}
        \rule{1pt}{\textheight}
        \hspace*{0.05\textwidth}
        {
        \parbox[c][][s]{0.75\textwidth}
        {
             \noindent
             \vspace{-12cm}
             \textsc{
             \setdefaultlanguage{russian}
             \setmainfont{Geometria}
             \Large
             Сборник Практик\\ [0.3\baselineskip]
             Лонгчен Ньингтик\\ [0.3\baselineskip]
             \\
             \vspace*{3cm}
             \\
             \normalsize
             Устье, 2015}
        }}
    \endgroup
    \setdefaultlanguage{russian}
    \setmainfont{Geometria}
}



% define images store

\graphicspath{{./images/}}

% start each section from new page

%\let\stdsection\section
%\renewcommand\section{\newpage\stdsection}

% define style for code listings

\lefthyphenmin=1
\hyphenpenalty=100
\tolerance=3000

%\newcommand\blankpage{
%    \null
%    \thispagestyle{empty}
%    \newpage}

\newcommand{\ru}{
    \setdefaultlanguage{russian}
    \setmainfont{Geometria}
}

\newcommand{\ti}{
    \setdefaultlanguage{tibetan}
    \setmainfont{DDC Uchen}
}

\newtoggle{russian@scriptlangequal}
\newtoggle{tibetan@scriptlangequal}

\newlength\tindent
\setlength{\tindent}{\parindent}
\setlength{\parindent}{0pt}
\renewcommand{\indent}{\hspace*{\tindent}}
