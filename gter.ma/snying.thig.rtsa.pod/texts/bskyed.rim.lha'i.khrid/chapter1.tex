\normalsize
\SectionNo{Лесница в Акаништху:\\ Наставления по Стадии Зарождения}
\\
Кланяюсь славному победителю Ваджрасаттве!

\begin{verse}
Чистый потенциал — совершенно чистый по природе\\
И свобод\-ный от завес,\\
Удинство проявлений и пустоты,\\
Чудесные мани\-фестации Мудрости.\\
Дабы это состояние трех Тел было легко достигнуто,\\
Я объясню то, чем руководствуются\\
В медитации йоги божества.
\end{verse}
\\
Живые существа блуждают в обманывающей их Сансаре ('khrul ba'i 'khorlo)
вследствие того, что их просветленный Потенциал временно нечист.
Стадия Зарождения Кьерим практики йидама устанавливает это
состояние как просветлённые Тело, Речь и Ум татхагаты, тайное
и непостижимое измерение абсолютного Пространства. Здесь я
обещаю полностью раскрыть этутему, которая называется
"Лестница в Акаништху".\\
\\
Текст состоит из трех основных разделов:\\
\\
\begin{tabular}{ll}
1 & Безошибочная причина — основа вступления в Кьерим.\\
2 & Безошибочное условие — путь медитации.\\
3 & Чистый результат — способ достижения \\
  & состояния единства.\\
\end{tabular}
\newpage
\Section{Безошибочная причина.\\Основа вступления в Стадию Зарождения}
\normalsize
\Vspace{1cm}
В "Этапах Пути" (lam rim) говорится:

\begin{verse}
Обладающий сокровищем,\\
Совершенным потоком и интересом,\\
Знанием тантры и активности,\\
Сущностными наставлениями и теплом,\\
— Таков учитель, наделённый восемью качествами.
\end{verse}

Здесь указывается, что необходимо приступать к изучению Стадии Зарождения под
руководством Ваджрного Учителя, обладающего восемью характеристиками, и служить ему
(или ей) тремя способами. Следующий этап — созревание, посред\-ством получения всех
необходимых посвящений:\\
\\
\begin{tabular}{ll}
1 & Благоприятствующее внешнее посвящение (phyi phan pa);\\
2 & Внутреннее посвящение силы (nang nus pa);\\
3 & Глубокое тайное посвящение (gsan ba zab mo).\\
\end{tabular}
\\
\\
\\
С этого момента необходимо поддерживать различные обеты самаи и обязательства
(dam tshig dang sdom pa). А также следует занимать сяпрактикой Стадии Зарождения йоги
божества.\\

\begin{siderules}
Эта часть содержит два раздела: 1) учения об очищении привычных тенденций, связанных счетырьмя видами рождения и
2) трех Самадхи — основе для процесса зарождения.\tablefootnote{Здесь и далее, выделенныеа бзацы текста соответствуют комментарию Патрула Чокьи Вангпо,
который называется "Разъяснение трудных мест в Стадии зарождения и Йоге божества"
(bskyed rim lha'i khrid kyi dka' gnad cung zad bshad pa bzhugs).}
\end{siderules}
\\
\\
\newpage

\SubSection{Четыре способа рождения и зарождения}
\Vspace{1cm}
\begin{siderules}
Вцелом, ключевым моментом (смыслом) всех путей Великой колесницы является очищение
природы Сансары — Истины страдания и ее источника, а также привнесение
результи\-рующего Состояния Будды на путь (lam du byed ра). И хотя Истина страдания
проявляется в различных формах, все они укоренены в рождении (skye ba).
Далее, все формы страдания могут быть разделены на рождение и смерть.
Поэтому есть две стадии, которые очищают этот двойной процесс рождения и смерти:
Стадия Зарождения и Стадия Завершения. Вся текстовая традиция Тайной мантры
связана с этими двумя практиками.
\end{siderules}

В "Славной магической сети" (dpal sgyu 'phrul drwa bа)\\ сказано:\\

\begin{verse}
Чтобы очистить четыре способа рождения,\\
Имеются четыре способа зарождения:\\
Очень сложный и сложный,\\
Простой и очень простой.
\end{verse}

\begin{siderules}
Как здесь показано, есть четыре типа рождения. Эти четыре, в свою очередь, связаны с
четырьмя типами практики Стадии Зарождения: краткой, средней, простран\-ной и очень
пространной (bsdus 'brin rgyas pa dang rgyas pa).
\end{siderules}

\newpage
\SubSubSection{Краткое описание предельной простоты}
\Vspace{1cm}
Те, кто обладают наивысшей проницательностью (высшими способностями) (dbang ро
yang rab), практикуют ритуал Стадии Зарождения предельной простоты (shin tu spros pa med
pa) посредством развития внутренней энергии ума (rtsal sems), который связан с общим
воззрением (lta ba spyi) наивысшей царской естественной Колесницы (rang bzhin theg mchog
rgyal ро). И мощный импульс (btsan thabs), который возникает из этого процесса, позволяет
упражняться в нераздельности Стадий Зарождения и Завершения. Не опираясь на слова,
природа ума предстает в своём изначальном состоянии, как совершенная форма божества.
Это происходит подобно мгнове\-нному появлению отражения в зеркале. Следующая цитата
описывает этот подход.

\begin{verse}
Божество — ты, и ты — божество.\\
Ты и божество возникаете вместе.\\
Поскольку Самайя и Мудрость недвойственны,\\
Нет необходимости ни приглашать божество,\\
Ни просить его воссесть.\\
Эманирующее из себя, самопосвящающее,\\
И самоосознающее —это Три Корня.\\
\end{verse}

В этой форме Стадии Зарождения природа божества неотъем\-лемо и совершенным
образом присутствует в иллюзорном проявлении Мудрости (yeshes sgyu mа).
Это очищает "чудесное рождение (rdzus skyes)". Объясняя далее, всеведущий Лонгчен\-па пишет:

\begin{verse}
Также как чудесное рождение возникает мгновенно,\\
Нет необходимости начинать с самого начала,\\
А затем медитировать\\
На Стадиях Зарождения и Завершения.\\
\end{verse}

\newpage
\SubSubSection{Промежуточный средний простой подход}
\Vspace{1cm}
Истина недвойственности Мудрости и Пространства позволяет тем, ктоо бладает
наивысшей проницательностью (способно\-стью) (dbang ро rab tu), практиковать простую
Стадию Зарож\-дения, используя мгновенный подход (cig char du 'jug pa). В такой практике
божества становятся совершенными в резуль\-тате припоминанияих сущности (snying ро dran
ра), и они незарождаются посредством слов. Можно также практиковать постепенное
вхождение (rim gyis 'jug), при котором божества естественным образом проявляются в
открытой сфере Саманта\-бхадри, в великой пустоте Пространства мудрости (shes rab kyi
dbyings). Таким образом, принимают единую и недвойственную иллюзорную форму
божества. В "Тантре естественного возник\-новения ригпа" (rang shar) об этом говорится так:

\begin{verse}
Что есть мгновенная практика?\\
Божество незарождают,но оно становится совершенным\\
В момент воспоминания сущности.\\
Как практикуют постепенное вхождение?\\
Посредством постепенного вхождения\\
В Пространство и Мудрость.\\
\end{verse}

Как здесь отмечено, опора — чудесный дворец (gzhal yas khang) и «опирающееся» —
божественная природа (lha'i rang bzhin), проявляющиеся совершенным
образом, либо посред\-ством произнесенияих (божеств) имён, либо посредством
при\-поминания их сущности. Такой вид Стадии Зарождения очища\-ет привычные тенденции
(bag chags), связанные с рождением из тепла и влаги (drodg sher skyes), а также
предполагает узнавание нераздельности Блаженства и Пустоты (bdes tong).

\newpage
\SubSubSection{Детальный и очень детальный подходы}
\Vspace{1cm}
Очень детальная Стадия Зарождения очищает рождение из яйца (sgong skye) и
предназначена для тех, чей ум склонен к концептуализации (rtog bcas). В этой связи
причинный херука возникает как результирующий Ваджрадхара (rdo rje 'dzin ра), когда
медитируют на различных аспектах ребёнка-себя и ребёнка-других. В учениях раздела
"Садхан Великих Восьми [Кагье]" (sgrub sde chen ро bka' brgyad) говорится, что существует
пять шагов, когда "других делают своим сыном" (bdag gi sras su gzhan bya ba):\\
\\
\begin{tabular}{ll}
1 & Из причинного семенного слога свет распространяется \\
  & во вне и втягивается обратно (spro bsdus), \\
  & порождая первичных Супругов (yab yum);\\
2 & Будды десяти направлений притягиваются (bkug) \\
  & и растворяются (bstim ра) в пространстве;\\
3 & Собираются живые существа и их завесы (sgrib) очищаются;\\
4 & Провозглашается (brjod ра) величие недвойственности;\\
5 & Божества вытягиваются (появляются?) (bton) \\
  & из пространства и помещаются (dgod ра) в мандалу.\\
\end{tabular}

\begin{siderules}
Если краткую и среднюю стадии легко понять, то в очень пространной стадии практики,
описанной выше, мы встречаемся с такими темами как "наш собственный сын" и "сын
других". Следующий отрывок рассматривает эту тему с точки зрения раздела тантр
(rgyud sde). В отношении "своего собственного сына" в "Магической сети" сказано:
\begin{verse}
Зная,что сам Плод является путем,\\
Естественным и без противоречий,\\
Медитируй на всех мандалах и тигле\\
Без исключения, как на своем сыне.\\
\end{verse}
\end{siderules}

\newpage 
"Делать себя сыном других" (gzhan gyi sras su bdag bya ba) состоит из восьми частей:\\
\\
\begin{tabular}{ll}
1 & Первичные супруги растворяются (bzhu) \\
  & и превращаются в причинный семенной слог;\\
2 & Супруги зарождаются из этого семенного слога;\\
3 & Из концепций (rtog tshogs) супруга зарождаются слоги;\\
4 & Из супруги излучается свет и \\
  & призывает божество (gsolbagdabра);\\
5 & Все мандалы растворяются в себе как мужские \\
  & и женские супруги, порождая гордость божества \\
  & Мудрости (ye shes par nga rgyal);\\
6 & Супруги соединяются (sbyor ba), \\
  & и в пространстве порождается мандала;\\
7 & 42 вида собственных концепций (rtogра) превращаются\\
  &  в божеств и расходятся вовне (phyir gdon ра);\\
8 & Божества мудрости приглашаются (spyan drang), \\
  & запечатываются (rgyas gdab ра) и т.д.\\
\end{tabular}

\begin{siderules}
И в связи с "сыном других" сказано:
\begin{verse}
Еслиу мы нераздельно вовлечены и слиты\\
Одним, обращенным с мольбой к другому,\\
Тогда какой смысл становиться сыном?\\
\end{verse}

Тогда как в "Тантре килы" (yang phurpa'i rgyud) относительно "сына других" сказано:
\begin{verse}
Подобно возникающему силой Будд,\\
В ритуале спонтанности\\
И из завершения ветвей ритуала,\\
Появляется мудрый — сын Будд.\\
\end{verse}
А также относительно "собственного сына" сказано:
\begin{verse}
Сын и царица возникают\\
Из явлений Сансары.\\
\end{verse}
Тогда как в разделе "Восьми великих садхан" сказано:
«Есть пять способов сделать других своим ребенком».
И хотя эта тема довольно ясная в каждом отдельном тексте, похоже, что нет ясного
объяснения, в котором было бы точно указано, что означает "собственный сын" и "сын
других". [Именно] Поэтому данная тема может казаться трудной для объяснения.
Однако, когда эта тема исследуется, первый и последний отрывок могут быть поняты как учащие
тому, что основа зарождения (bskyed gzhi) является Дхарматой самости — Сугатагарбхой.
Таким образом, визуализируя форму этого причинного Держате\-ля ваджры (Ваджрадхары),
"другие" — все цепляния и прояв\-ления (snang zheri), которые мы воспринимаем в Сансаре и
Нирване, вместе с промежуточным состоянием (bar srid) — собираются, а затем
"делают сына", извлекая его и злона. Этот процесс называется "делать других
своим собственным сыном".\\
\\
Когда "делают себя сыном других", то все проявления и восприятия Сансары и Нирваны
("другие") визуализируют\-ся как главное божество мандалы (Ваджрадхара). Это превра\-щает
все накопленные цепляния по отношению к скандхам, дхату и аятанам в семенной слог,
который затем становится "сделанным сыном" в результате извлечения из лона. Таким
образом становятся "сыном других".\\
\\
В среднем отрывке сказано, что сансарические существа являются полностью и совершенно
просветленными, как Будды трех времен. И здесь живые существа рассматрива\-ются как
"сами", а Будды как "другие". Делая себя "ребенком Будд", появляются в результате
зарождения себя как глав\-ного божества мандалы в ходе практик совершенствования,
таких как ритуал спонтанности (grub pa'i choga) и пять ветвей ритуала (cho ga'i yan lag lnga).\\
\newpage
"Делать других своим собственным ребенком", с другой стороны, относится к процессу,
в котором все проявляющееся как собрание божеств в круге мандалы является "другим",
а все сансарические существа — "собой". Последовательность, в которой каждый из них возникает,
гармонизируется (sgo bstun) в этом подходе, что ведет к союзу мужского и женского
супругов, а также к последующему развитию через извлечение из лона. Первые два из этих
объяснений представляют эту концепцию с точки зрения основы зарож\-дения, тогда как
последующее — с точки зрения процесса зарождения (bskyed tshul).\\
\\
Здесь может возникнуть вопрос: являются ли эти два подхода представления себя сыновьями
хоть сколько-нибудь различными. Однако, в "Колеснице всеведущих" (rnam mkhyen shing
rta) есть объяснение, которое связывает все три. Поэтому я призываю разумных существ
тщательно исследовать это объяснение и открыть врата к этому ясному объяснению.
\end{siderules}
\newpage

Согласно тексту "Великая волшебная сеть" (rgyu 'phrul drwaba chenро) начинать
нужно с принятия Прибежища и зарождения Бодхичитты. Происходящее далее и связанное
с рождением из яйца, осуществляется в два этапа. Сначала, мгновенно представьте себя
изначальными супругами (yab yum), а затем пригласите в пространство перед собой мандалу,
на которую вы медитируете. Далее совершайте подношения, восхваления, молитвы,
постоянные раскаяния (mchod bstod gsol gdab rgyun bshags) и т.д. После этого используйте
ВАДЖРА МУ, чтобы снова пребывать в медитативном равновесии (mnyam par gzhag) в
сфере Пустоты. Этот последний пункт характерен исключительно для данного подхода.
Зная это, следующее представление может быть использовано как в предельно-развернутом
подходе, так и всредне-детализированной Стадии Зарождения, которая очищает рождение
из утробы.\\
\\
В этих случаях основной упор делается на искусных методах Стадии Зарождения.
Также как буйствующий слон может быть управляем крюком или толпой, есть два подхода к этой
активности. Форма божественной обители и самих божеств очищают два элемента: опору и
то, что опирается. Первое — [опора] относится к внешней вселенной, а второе 
[то, что опирается] — к физическому телу. Существованиеих обоих 
[вселенная-«опора» и мы — существа-«опирающееся»] опирается на различные
привычные тенденции (bag chags). Следующий метод успокаивает склонность ума
испускать внешние объекты (yul la 'phro ba) и заключается в установлении
столба (rtod ра) глубокого медитативного погружения (ting ge 'dzin).\\
\\
Этот подход позволяет соединиться с сущностью очищения, совершенствования и
созревания (dag rdzogs smin). А точнее, поскольку это соответствует характеристикам
Сансары, существование (srid ра) очищается и улучшается. И поскольку это сходно с путём
Нирваны, результат совершенен в основе (gzhi la rdzogs). И, наконец, оба они приводят к
созреванию Стадии Завершения. Это общее понимание имеет чрезвычайно важное значение.\\
\\
\begin{siderules}
\end{siderules}

\SubSection{Три самадхи}
\Vspace{1cm}
В самом начале практика Стадии Зарождения проходит через [этапы] трех самадхи:

\begin{tabular}{ll}
1 & самадхи Таковости;\\
2 & самадхи Ясного Света;\\
3 & причинное самадхи.\\
\end{tabular}

Далее описывается сущностная природа самадхи таковости (de bzhin nyid kyi ting ge 'dzin
gyi ngo bo): Внутри и вне себя, ум независит от какой либо основы. Нет корня, из которого
он вырастает. Он не существует в какойлибо онтологической крайности (dngos ро' mtha'). Он
не мужского, не женского и несреднего рода. У него нет цвета, структуры или формы.
Однако, поскольку он по своей природе является Ясным Светом (rang bzhin gyi 'od gsal), он
такженеявляетсяипростымничто.ВрезультатевоспоминанияМудростиДхарматы,как
онаестьнасамомделе(chosnyidjiItarba'iyeshes),существованиесмертиочищаетсяв
Дхармакайю.Уверенностьвтом,чтовещипостоянны,очищаетсятакжекакисферабез
формы(gzugsmedkhams).Такимобразом,этоназывают"самадхитаковости".
СамадхиЯсногоСвета('odgsal)называетсятак,посколькуестественноесияниеэтого
великогопустогоСветаявляетсяуравнивающимиобъединяющимсостраданиемпо
отношениюковсемживымсуществам,атакжеоноосвобождаетотнигилистических
воззренийисферуформ gzugskyikhams).Крометого,непостижимаяпроявляющаяся
энергия(rolrtsal)этойраспространяющейМудрости(mchedраЧyeshes)готовитпочвудля
превращенияпромежуточногосостояния-бардовСамбхогакайю.Этотакназываемое"все
освещающее(кипtusnangba)самадхи".
Изэтогосамоосознавание(rigраnyid)появляетсязатемвформеслоговА,ХУМили
ХРИ.Этопроцессизвестенкак"причинныйметод",атакже"кореннойметод".Причинное
самадхиочищаетсознаниенынешнегомоментасуществования,котороеготововойтив
новоеместопребывание.Крометогооноочищаетсферужеланий('dodkhams)иприводитк
созреваниюрождениевНирманакае.Таковопричинноесамадхи.

\SubSection{Зарождение поддерживающей и поддерживаемой мандал}

ЗаложивосновудляСтадииЗарожденияэтимитремявидамисамадхи,далееможно
приступатькзарождениюподдерживающегоиподдерживаемого.Точнаяприродаэтого
процессаизложенавобширномсобраниитантртрадициираннихпереводовньингма,всвязи
стриадойЗарождения,ЗавершенияиВеликогоСовершенства,втакомкак"Великаятайная
сущность"(<dpalgsangЬаЧsnyingро).Вэтомподходепричинныйиликореннойметод
приводиткосновополагающемуспособувизуализациидворцабожестваитрона.Иэто,в
своюочередь,ведёткневообразимому(bsamyas)способумедитациинавсей
поддерживаемоймандале.

Крометого,медитатациянаформечудесногодворцавбескрайнемпространстве
благословляет(byinbrlab)нечистуюприродусосуда(т.е.этогомира)какАкаништху.
ТридцатьсемьфакторовПросветления(byangchubkyichossumcurtsabdun)-этото,что
устанавливаетсущностнуюсвязь(sbyorЬаЧngobo)междуосновой,путем,Плодом,
результатомичистотой.Природаэтойсвязиобъясняетсявдевятойглаветекста"Великая
колесница(shingrtachenто)".

Здесь,однако,мысосредоточимсявосновномнастадияхвизуализацииподдерживаемого
божества(brtenра),связываяспособ,которымвозникаетваджрноетело,соСтадией
Зарождения.Данныйпроцессочищаетоснову(объект)очищения(sbyanggzhi).
Втантре"ХерукаГалпо(heruкаgalро)"сказано:

Первое-ШуньятаиБодхичитта,
Второе-возникновениеСемени,
Третье-совершеннаяформа
Четвёртое-установление[коренного]Слога.

Здесьговоритсяотом,чтосмертьипромежуточноесостояниеочищаютсяШуньятой
(Пустотностью)иБодхичиттой.Сознаниевформебестелесногодуха(driza),готовоевойти
всоединениесемениияйцеклетки,очищаетсясобираниемсемени(sabonbsduba).Затем
постепенноразвиваетсятело,котороесгущаетсядесятьюветрами(энергиями)(гlungЪси).
Этотпроцессочищаетсясовершеннойформой(gzugsrdzogsра).Будучирождёнными,
чувственныеспособности(органывосприятия)(dbangро)проявляютактивность(sadра)по
отношениюксвоимобъектам.Иэто,всвоюочередь,очищаетсяустановлениемслога(yige
'godра)ит.д.Данноеобъяснениеимеетотношениексимволамчетырёхманифестаций
просветления(mngonbyangbzhi'brda).
Однакоздесьбудетданообъяснениесточкизренияпостепенногоразвития,что
соответствуетобщемувзгляду"СобранияТантр(rgyudsde)".Согласноотцовскимтантрам
зарождениепроисходитпосредствомритуалатрёхваждр,втовремякаксогласно
материнскимтантрамэтопроисходитпосредствомпятиманифестацийпросветления.В
обеихэтихтрадициях,основаочищениясвязанасчистотойпроцессаочищения.

