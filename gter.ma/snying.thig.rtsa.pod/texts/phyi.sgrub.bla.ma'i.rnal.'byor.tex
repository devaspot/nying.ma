% HEVEA % Copyright (c) 2015 Synrc Research Center

\usepackage[english,russian]{babel}
\usepackage{fontspec}
\usepackage{tablefootnote}
\usepackage{polyglossia}
\usepackage{hyphenat}
\usepackage[top=8mm, bottom=12.4mm,
            inner=5mm,outer=8mm,
            paperwidth=122mm, paperheight=180mm]{geometry}

\fontencoding{T1}
\lefthyphenmin=1
\hyphenpenalty=100
\tolerance=3000
\newcommand{\ru}{ \setdefaultlanguage{russian} \setmainfont{Geometria} }
\newcommand{\ti}{ \setdefaultlanguage{tibetan} \setmainfont{DDC Uchen} }
\newtoggle{russian@scriptlangequal}
\newtoggle{tibetan@scriptlangequal}


% HEVEA \begin{document}
% nyingma_author =Maxim Sokhatsky=
% HEVEA \title{Драгоценность Исполняющая Желания: Внешняя Гуру Йога}

\normalsize
\Section{Джигме Лингпа.\\Драгоценность Исполняющая Желания:\\Садхана Внешней Гуру-йоги}

{\ti འུྃ༔ ཀློང་ཆེན་སྙིང་གི་ཐིག་ལེ་ལས༔\\
ཕྱི་སྒྲུབ་བླ་མའི་རྣལ་འབྱོར་ཡིད་བཞིན་ནོར་བུ་བཞུགས༔}\\
\\

{\ti ༔རིག་པ་དབྱིངས་ཀྱི་ལྷ་ལ་འདུད༔}\\
\\
\ru Простираюсь перед манифестацией обшиного пространства самоосознавания!\\
\\
{\ti བླ་མ་ཕྱི་ལྟར་སྒྲུབ་པ་ནི༔\\
མ་ངེས་གནས་སྐབས་ཐམས་ཅད་དུ༔\\
ལུས་ནི་རྣམ་སྣང་ཆོས་བདུན་དང་༔\\
མོས་གུས་དྲག་པོའི་ཀུན་སློང་གིས༔\\
ཤེས་པ་གཞན་དུ་མ་ཡེངས་པར༔\\
གཞི་རྫོགས་རྟེན་བསྐྱེད་འདི་ལྟར་གསལ༔}\\
\\
\ru Внешний аспект Учителя — полностью неопределен.\\
Тело Учителя — поза Вайрочаны.\\
Глубокая и преданная мотивация\\
Предстает в образе Завершенной Основы\\
Без отвелений — самоосознавания.\\
\\
{\ti ཨེ་མ་ཧོ༔\\
རང་སྣང་ལྷུན་གྲུབ་དག་པ་རབ་འབྱམས་ཞིང་༔\\
བཀོད་པ་རབ་རྫོགས་ཟངས་མདོག་དཔལ་རིའི་དབུས༔\\
རང་ཉིད་ོ་བོ་མཚོ་རྒྱལ་ལ་རྣམ་པ་རྣལ་འབྱོར་མར་གསལ་བ་རྗེས་སུ་འཛིན་པའི་རྟེན་འབྲེལ་གྱི་མན་ངག\\
རྗེ་བཙུན་རྡོ་རྗེ་རྣལ་འབྱོར་མ༔\\
ཞལ་གཅིག་ཕྱག་གཉིས་དམར་གསལ་གྲི་ཐོད་འཛིན༔\\
ཞབས་གཉིས་༼༡༽དོར་སྟབས་སྤྱན་གསུམ་ནམ་མཁར་གཟིགས༔\\
སྤྱི་བོར་པདྨའབུམ་བརྡལ་ཉི་ཟླའི་སྟེང་༔\\
སྐྱབས་གནས་ཀུན་འདུས་རྩ་བའི་བླ་མ་དང་༔\\
དབྱེར་མེད་མཚོ་སྐྱེས་རྡོ་རྗེ་སྤྲུལ་པའི་སྐུ༔\\
དཀར་དམར་མདངས་ལྡན་གཞོན་ནུའི་ཤ་ཚུགས་ཅན༔\\
ཕོད་ཁ་ཆོས་གོས་ཟབ་བེར་འདུད་མ་གསོལ༔\\
ཞལ་གཅིག་ཕྱག་གཉིས་རྒྱལ་པོ་རོལ་པའི་སྟབས༔\\
ཕྱག་གཡས་རྡོ་རྗེ་གཡོན་པས་ཐོད་བུམ་བསྣམས༔\\
དབུལ་འདབ་ལྡན་པདྨི་མཉེན་ཞུ་གསོལ༔\\
མཆན་ཁུང་གཡོན་ན་བདེ་སྟོང་ཡུམ་མཆོག་མ༔\\
སྦས་པའི་ཚུལ་གྱིས་ཁ་ཊྭཾ་རྩེ་གསུམ་བསྣམས༔\\
འཇའ་ཟེར་ཐིག་ལེ་འོད་ཕུང་ཀློང་ན་བཞུགས༔\\
ཕྱི་འཁོར་འོད་ལྔའི་དྲྭ་བས་མཛེས་པའི་ཀློང་༔\\
སྤྲུལ་པའི་རྗེ་འབངས་ཉི་ཤུ་རྩ་ལྔ་དང་༔\\
རྒྱ་བོད་པན་གྲུབ་རིག་འཛིན་ཡི་དམ་ལྷ༔\\
མཁའ་འགྲོ་ཆོས་སྐྱོང་དམ་ཅན་སྤྲིན་ལྟར་གཏིབས༔\\
གསལ་སྟོང་མཉམ་གནས་ཆེན་པོའི་ངང་དུ་གསལ༔}\\
\\
\ru ЭМАХО\\
Измерение Бесконечной Чистоты Спонтанного Самовозникновения\\
Место Завершенния — Центр Высшей Славная Медноцветной Горы\\
\\
{\ti ཧཱུྃ༔ ཨོ་རྒྱན་ཡུལ་གྱི་ནུབ་བྱང་མཚམས༔\\
པདྨགེ་སར་སྡོང་པོ་ལ༔\\
ཡ་མཚན་མཆོག་གི་དངོས་གྲུབ་བརྙེས༔\\
པདྨའབྱུང་གནས་ཞེས་སུ་གྲགས༔\\
འཁོར་དུ་མཁའ་འགྲོ་མང་པོས་བསྐོར༔\\
ཁྱེད་ཀྱི་རྗེས་སུ་བདག་བསྒྲུབ་ཀྱིས༔\\
བྱིན་གྱིས་བརླབ་ཕྱིར་གཤེགས་སུ་གསོལ༔\\
གུ་རུ་པདྨསིདྡྷ་ཧཱུྃ༔\\
\\
ཧྲཱི༔ བདག་ལུས་ཞིང་གི་རྡུལ་སྙེད་དུ༔\\
རྣམ་པར་འཕྲུལ་པས་ཕྱག་འཚལ་ལོ༔\\
དངོས་བཤམས་ཡིད་སྤྲུལ་ཏིང་འཛིན་མཐུས༔\\
སྣང་སྲིད་མཆོད་པའི་ཕྱག་རྒྱར་འབུལ༔\\
སྒོ་གསུམ་མི་དགེའི་ལས་རྣམས་ཀུན༔\\
འོད་གསལ་ཆོས་སྐུའི་ངང་དུ་བཤགས༔\\
བདེན་པ་གཉིས་ཀྱིས་བསྡུས་པ་ཡི༔\\
དགེ་ཚོགས་ཀུན་ལ་རྗེས་ཡི་རང་༔\\
ཐེག་གསུམ་ཆོས་འཁོར་བསྐོར་བར་བསྐུལ༔\\
ཇི་སྲིད་འཁོར་བ་མ་སྟོངས་བར༔\\
མྱ་༼༢༽ངན་མི་འདའ་བཞུགས་གསོལ་འདེབས༔\\
དུས་གསུམ་བསགས་པའི་དགེ་རྩ་ཀུན༔\\
བྱང་ཆུབ་ཆེན་པོའི་རྒྱུ་རུ་བསྔོ༔\\
རྗེ་བཙུན་གུ་རུ་རིན་པོ་ཆེ༔\\
ཁྱེད་ནི་སངས་རྒྱས་ཐམས་ཅད་ཀྱི༔\\
ཐུགས་རྗེ་བྱིན་རླབས་འདུས་པའི་དཔལ༔\\
སེམས་ཅན་ཡོངས་ཀྱི་མགོན་གཅིག་པུ༔\\
ལུས་དང་ལོངས་སྤྱོད་བློ་སྙིང་བྲང་༔\\
ལྟོས་པ་མེད་པར་ཁྱེད་ལ་འབུལ༔\\
འདི་ནས་བྱང་ཆུབ་མ་ཐོབ་པར༔\\
སྐྱིད་སྡུག་ལེགས་ཉེས་མཐོ་དམན་ཀུན༔\\
རྗེ་བཙུན་ཆེན་པོ་པད་འབྱུང་མཁྱེན༔\\
བདག་ལ་རེ་ས་གཞན་ན་མེད༔\\
ད་ལྟའི་དུས་ངན་སྙིགས་མའི་འགྲོ༔\\
མི་བཟད་སྡུག་བསྔལ་འདམ་དུ་བྱིང་༔\\
འདི་ལས་སྐྱོབས་ཤིག་མ་ཧཱ་གུ་རུ༔\\
དབང་བཞི་བསྐུར་ཅིག་བྱིན་རླབས་ཅན༔\\
རྟོགས་པ་སྤོར་ཅིག་ཐུགས་རྗེ་ཅན༔\\
སྒྲིབ་གཉིས་སྦྱོངས་ཤིག་ནུས་མཐུ་ཅན༔\\
ནམ་ཞིག་ཚེ་ཡི་དུས་བྱས་ཚེ༔\\
རང་སྣང་རྔ་ཡབ་དཔལ་རིའི་ཞིང་༔\\
ཟུང་འཇུག་སྤྲུལ་པའི་ཞིང་ཁམས་སུ༔\\
གཞི་ལུས་རྡོ་རྗེ་རྣལ་འབྱོར་མ༔\\
གསལ་འཚེར་འོད་ཀྱི་གོང་བུ་རུ༔\\
གྱུར་ནས་རྗེ་བཙུན་པད་འབྱུང་དང་༔\\
དབྱེར་མེད་ཆེན་པོར་སངས་རྒྱས་ཏེ༔\\
བདེ་དང་སྟོང་པའི་ཆོ་འཕྲུལ་གྱི༔\\
ཡེ་ཤེས་ཆེན་པོའི་རོལ་པ་ལས༔\\
ཁམས་གསུམ་སེམས་ཅན་མ་ལུས་པ༔\\
འདྲེན་པའི་དེད་དཔོན་དམ་པ་རུ༔\\
རྗེ་བཙུན་པདྨ་དབུགས་དབྱུང་གསོལ༔\\
གསོལ་བ་སྙིང་གི་དཀྱིལ་ནས་འདེབས༔\\
ཁ་ཙམ་ཚིག་ཙམ་མ་ཡིན་ནོ༔\\
བྱིན་རླབས་ཐུགས་ཀྱི་ཀློང་ནས་སྩོལ༔\\
བསམ་དོན་འགྲུབ་པར་མཛད་དུ་གསོལ༔\\
ཨོཾ་ཨཱཿཧཱུྃ་བཛྲ་གུ་རུ་པདྨསིདྡྷ་ཧཱུྃ༔}\\
\\
\ru ОМ А ХУМ БАДЗРА ГУРУ ПАДМА СИДДХИ ХУМ\\
\\
{\ti ཞེས་ཁ་ཞེ་ངོས་ལྐོག་མེད་པར་བློས་ལིངས་བསྐུར་གྱིས་བཟླས་པ་ཆུ་བོའི་རྒྱུན་བཞིན་བཟླ་བ་དང་མཉམ་དུ་དབང་བཞི་ལེན་པའི་དམིགས་པ་བྱ༔\\
དམིགས་པ་མི་གསལ་བ་དང་ཁམས་འདུས་པ་སོགས་བྱུང་ན༔\\
སྣང་སྲིད་སྣོད་བཅུད་ཐམས་ཅད་ཨོ་རྒྱན་ཆེན་པོའི་སྐུ་གསུང་ཐུགས་ཀྱི་རོལ་པར་དག་སྣང་སྦྱངས་ཏེ་གསོལ་བ་ཕུར་ཚུགས་སུ་བཏབ༔\\
ཉམས་སྤྲོ་བ་དང་རིག་པ་དངས་བའི་ཚེ་སྣང་བ་གཏད་མེད་དུ་བསྒྱུར་ནས་ད་ལྟའི་ཤེས་པ་སྐད་ཅིག་མ་ཡེ་རེ་བ་སྐྱང་༔\\
སྡུག་བསྔལ་གཏམ་ངན་ན་ཚ་སོགས་བྱུང་ན་བདག་གི་ལས་ངན་འཛད་པའི་ཐབས་སུ་བླ་མས་གནང་བ་ཡིན་པས་ང་རེ་དགའ་སྙམ་དུ་ཡིད་དང་ས་སྤྲོ་སེང་ངེ་བ་གྱིས་ལ་རླུང་རོ་འབུད་ཅིང་གསོལ་འདེབས་ལ་འབད༔\\
གདོན་ག་གེགས་ཀྱི་འཚེ་བ་བྱུང་ན་དེ་ལ་སྙིང་རྗེ་དུད་དུད་བ་བསྐྱེད་ལ་ཁོའི་སེམས༔\\
རང་གི་རིག་པ་གུ་རུའི་ཐུགས་གསུམ་དབྱེར་མེད་དུ་བསྲེས་ལ་གནོད་བྱེད་ཀྱི་རྣམ་རྟོག་རྩ་གདར་གཅད་དེ་བཛྲ་གུ་རུ་ལ་འབད༔\\
རྟག་པར་གང་དུ་འདུག་པའི་གནས་ཟངས་མདོག་དཔལ་རི་དངོས་ཡིན་སྙམ་པ་ལས་ཐ་མལ་དུ་༼༣༽མི་བལྟ༔\\
དེས་ཞིང་ཁམས་འབྱོངས་པ་ཡིན་ནོ༔\\
འགྲོ་བའི་ཚེ་སངས་རྒྱས་ཀྱི་ཞིང་ཁམས་རྡུལ་ཕྲ་རབ་ལ་བསྐོར་བ་བྱེད་པར་བསམ༔\\
ཅི་ཟ་ཅི་འཐུང་གི་ཕུད་བདུད་རྩིའི་རང་བཞིན་དུ་བསམས་ཏེ་སྤྱི་བོའི་བླ་མ་ལ་འབུལ༔\\
ཉལ་བའི་ཚེ་སྤྱི་བོ་ཚངས་པའི་བུ་གནས་མར་ལ་ཡེར་གྱིས་བྱོན་ནས༔\\
རང་གི་སྙིང་པདྨའདབ་བཞིའི་རྣམ་པར་གནས་པའི་ནང་དུ་བྱོན༔\\
དེའི་འོད་ཟེར་གྱིས་རང་ལུས་སྣོད་བཅུད་དང་བཅས་པ་འོད་གསལ་དུ་སྦྱངས་པའི་སྣང་བ་ནང་གསལ་ཐིམ་ལམ་རྨུགས་པའི་ངང་དུ་ཁམས་གསོ༔\\
གལ་ཏེ་སད་པ་ན་འཕྲོ་རྒོད་རྒྱ་འབྱམས་དང་རྨི་ལམ་སོགས་ལ་བསམ་གཞིག་གི་འཕྲོ་བཅད་དེ་འོད་གསལ་གྱི་ངང་མདངས་ཁྱབ་གདལ་དུ་བསྐྱང་༔\\
ནངས་པར་ལྡངས་པའི་ཚེ་ཆུ་ལས་ཉ་འཕར་བ་བཞིན༔\\
གཞི་ལུས་ཞིང་ཁམས་ཀྱི་བཀོད་པ་དང་བཅས་པ་གསལ་བཞིན་པའི་ངང་ལས་སྙིང་གའི་བླ་མ་དབུ་མའི་ལམ་བརྒྱུད་སྤྱི་བོའི་སྟེང་གི་བར་སྣང་ལ་ཡེ་རེ་ཁྱིལ་ལེར་དགྱེས་བཞིན་དུ་བཞུགས་པར་བསམ༔\\
མདོར་ན་དུས་དང་རྣམ་པ་ཐམས་ཅད་དུ་ཕྱི་སྣོད་ཀྱི་འཇིག་རྟེན་ཟངས་མདོག་དཔལ་རིའི་ཕོ་བྲང་༔\\
ནང་བཅུད་ཀྱི་སེམས་ཅན་དཔའ་བོ་མཁའ་འགྲོའི་ཚོམ་བུ༔\\
གསང་བ་སེམས་ཀྱི་འཕྲོ་འདུ་རང་གྲོལ་བྱ་ལམ་རྗེས་མེད་དུ་ཤེས་པའི་ངང་ནས་སྙིང་པོ་བཛྲ་གུ་རུ་ཁོ་ན་ལ་འབད་ཅིང་༔\\
ཐུན་མཚམས་སུ་ཟངས་མདོག་དཔལ་རིའི་སྨོན་ལམ་འདོན་ཞིང་བསྔོ་བས་རྒྱས་འདེབས༔\\
དུས་དང་དུས་མ་ཡིན་པ་སོགས་མདོར་ན་ནམ་འཆི་བ་ལ་བབས་པའི་ཚེ་གཞི་ལུས་རྡོ་རྗེ་རྣལ་འབྱོར་མའི་རྣམ་པ་འོད་ཀྱི་གོང་བུར་གྱུར་ནས་སྤྱི་བོའི་བླ་མ་ལ་བསྟིམས༔\\
བླ་མ་རང་སེམས་གཉིས་སུ་མེད་པའི་ངང་ལ་ཤེས་པ་སྤྲོ་བསྡུ་དང་བྲལ་བར་བཞག་པ་ནི་འཕོ་བ་ཐམས་ཅད་ཀྱི་རྒྱལ་པོ་ཡིན་ལ༔\\
འདི་ལྟ་བུའི་མན་ངག་ནི་བསྐྱེད་རིམ་གྱི་གནད༔ རྫོགས་རིམ་གྱི་བཅུད༔ གནས་ལུགས་ཀྱི་མཚང་༔\\
མན་ངག་གི་མདུད་ཡིན་པས་ཆོས་སྒོ་བརྒྱད་ཁྲི་བཞི་སྟོང་ཞལ་བསྡུར་ཀྱང་གནད་འདི་ལས་ཟབ་པ་འབྱུང་མི་སྲིད་པའི་ཟབ་རྒྱ་ཡིན་པས་སྐལ་ལྡན་རྣམས་ཀྱི་སྙིང་ལ་གཅེས་པར་ཟུངས་ཞིག༔\\
ཨེ་མ་ཀློང་ཆེན་སྙིང་གི་ཐིག་ལེ་ལས༔\\
ཕྱི་སྒྲུབ་བླ་མའི་རྣལ་འབྱོར་སྐོར༔\\
ཟབ་གཉན་བྱིན་རླབས་ཚན་ཁ་ཆེ༔\\
འོད་གསལ་དབུ་མའི་ཕོ་བྲང་དུ༔\\
ཤེས་རིག་པདྨཐོད་ཕྲེང་གིས༔\\
བྱིན་རླབས་བརྡ་བརྒྱུད་ཚུལ་དུ་གདམས༔\\
འགྲོ་དོན་དུས་ལ་བབ་པའི་ཚེ༔\\
ཐུགས་གཏེར་ནམ་མཁའ་མཛོད་ཀྱི་སྒོ༔\\
རྟེན་འབྲེལ་ཟབ་མོའི་རྐྱེན་གྱིས་ཕྱེད༔\\
ཤིན་ཏུ་ཟབ་པའི་མཐར་ཐུག་གོ༔\\
ས་མ་ཡ༔\\
རྒྱ་རྒྱ་རྒྱ༔\\
\\
༈ དབང་བཞི་ལེན་པའི་དམིགས་པ་ནི༔}\\
\\
\ru Получение Четырех Посвящений\\
\\
{\ti བླ་མ་ལ་གདུང་བ་དྲག་པོའི་རྐྱེན་གྱིས་བདག་ཉིད་ལ་བརྩེ་བས་རྗེས་སུ་དགོངས་ཏེ།\\
གུ་རུའི་སྨིན་མཚམས་ན་ཨོཾ་ཡིག་ཆུ་ཤེལ་ལྟ་བུར་འཚེར་བ་ལས་འོད་ཟེར་འཕྲོས།\\
རང་གི་སྤྱི་བོ་ནས་ཞུགས།\\
ལུས་༼༤༽ཀྱི་ལས་དང་རྩའི་སྒྲིབ་པ་དག\\
།སྐུ་རྡོ་རྗེའི་བྱིན་རླབས་ཞུགས།\\
བུམ་པའི་དབང་ཐོབ།\\
བསྐྱེད་རིམ་གྱི་སྣོད་དུ་གྱུར།\\
རྣམ་སྨིན་རིགས་འཛིན་གྱི་ས་བོན་ཐེབས།\\
སྤྲུལ་སྐུའི་གོ་འཕང་ཐོབ་པའི་སྐལ་པ་རྒྱུད་ལ་བཞག\\
།མགྲིན་པ་ན་ཨཱཿཡིག་པདྨརཱ་ག་ལྟར་འབར་བ་ལས་འོད་ཟེར་འཕྲོས།\\
རང་གི་མགྲིན་པ་ནས་ཞུགས།\\
ངག་གི་ལས་དང་རླུང་གི་སྒྲིབ་པ་དག\\
།གསུང་རྡོ་རྗེའི་བྱིན་རླབས་ཞུགས།\\
གསང་བའི་དབང་ཐོབ།\\
བཟླས་བརྗོད་ཀྱི་སྣོད་དུ་གྱུར།\\
ཚེ་དབང་རིགས་འཛིན་གྱི་ས་བོན་ཐེབས།\\
ལོངས་སྤྱོད་རྫོགས་པའི་གོ་འཕང་གི་སྐལ་བ་རྒྱུད་ལ་བཞག\\
།ཐུགས་ཀའི་ཧཱུྃ་ཡིག་ནམ་མཁའི་མདོག་ཅན་ལས་འོད་ཟེར་འཕྲོས།\\
རང་གི་སྙིང་ག་ནས་ཞུགས། ཡིད་ཀྱི་ལས་དང་ཐིག་ལེའི་སྒྲིབ་པ་དག\\
།ཐུགས་རྡོ་རྗེའི་བྱིན་རླབས་ཞུགས།\\
ཤེས་རབ་ཡེ་ཤེས་ཀྱི་དབང་ཐོབ།\\
བདེ་སྟོང་ཙཎྜ་ལཱིའི་སྣོད་དུ་གྱུར།\\
ཕྱག་རྒྱའི་རིགས་འཛིན་གྱི་ས་བོན་ཐེབས།\\
ཆོས་སྐུའི་གོ་འཕང་གི་སྐལ་བ་རྒྱུད་ལ་བཞག\\
།སླར་ཡང་ཐུགས་ཀའི་ཧཱུྃ་ལས་ཧཱུྃ་ཡིག་གཉིས་པ་ཞིག་སྐར་མདའ་འཕངས་པ་བཞིན་དུ་ཆད།\\
རང་སེམས་དང་ཐ་དད་མེད་པར་འདྲེས།\\
ཀུན་གཞིའི་ལས་དང་ཤེས་བྱའི་སྒྲིབ་པ་སྦྱངས།\\
ཡེ་ཤེས་རྡོ་རྗེའི་བྱིན་རླབས་ཞུགས།\\
ཚིག་གིས་མཚོན་པ་དོན་དམ་གྱི་དབང་ཐོབ།\\
ཀ་དག་རྫོགས་པ་ཆེན་པོའི་སྣོད་དུ་གྱུར།\\
ལྷུན་གྲུབ་རིགས་འཛིན་གྱི་ས་བོན་ཐེབས།\\
མཐར་ཐུག་གི་འབྲས་བུ་ངོ་བོ་ཉིད་སྐུའི་སྐལ་བ་རྒྱུད་ལ་བཞག་གོ\\
།དེ་རྗེས་གུ་རུའི་སྐུ་གསུང་ཐུགས་དང་རང་གི་སྒོ་གསུམ་དུ་མ་རོ་གཅིག་ཏུ་བསྲེས་པའི་ངང་ལ་ལྟ་བའི་ངང་མདངས་བསྐྱང་ཞིང་གསོལ་འདེབས་བསྙེན་པ་ལ་འབད།\\
\\
འདི་ནི་བླ་མ་ཆོས་སྐུའི་ཞལ་ལྷ་བ་ཞེས་བྱའོ༔}\\
\\

% HEVEA \end{document}
